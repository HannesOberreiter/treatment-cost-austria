\pdfbookmark[0]{Abstract}{Abstract}
\addcontentsline{toc}{chapter}{Abstract}
\chapter*{Abstract}
\label{sec:abstract}
\vspace*{-10mm}

Exploratory analysis on the annual colony loss online survey in Austria over the three winters 2018/19, 2019/20 and 2020/21. Main focus is to identify possible reasons for choosing a treatment method against the parasitic mite \textit{Varroa destructor}. In addition to evaluate effectiveness and expenses of these methods. This is done by analysing the questions \enquote{Which main motivations are most likely to apply to you when choosing the varroa control method (max. 5)?} and \enquote{Estimated treatment expenses per colony without labour cost} in combination with the applied treatment methods in combination and their effectiveness compared to all other reported ones. Response rate of 3.5-4.0\% of beekeepers over the survey years. Reported mean expenses on treatment of EUR~11.5 and median EUR~10 per colony. Reported expenses per colony show high variance also inside the same treatment methods. Rough extrapolation of the total Austrian varroa mite agent market shows that the yearly expenses could be in the range of EUR~2.9 to 4.9 million a year. Participants main motivation for choosing a treatment method are 'Efficacy', 'Side effects on bees' and 'Residues'. The main motivations indicate that beekeepers select methods based on efficacy and intrinsic motives for their honey bees well being and are aware of the residue problematic. Smaller hobbyist beekeepers have different motivations than bigger operations, which are more economic driven. Most treatment combinations show in each year overlapping confidence intervals for observed loss rates, indicating similar effectiveness. Survey treatment methods against \textit{Varroa d.} is dominated by formic acid application in summer followed by the absolute dominance of oxalic acid treatment in winter. Smaller operations ($\leq$ 25 colonies) spend more on treatment per colony than bigger operations. Migratory and certified organic beekeepers spend less on average. A multiple regression model explanatory ability of the treatment methods and operation size on the treatment expenses is low, with an accuracy on the training dataset of 35\%. The most used combination formic acid long term treatment in summer followed by oxalic acid trickling in winter showed low to intermediate loss rates, but high to intermediate expenses compared to other treatment combinations. A combination of oxalic acid sublimation in summer and winter and the same but with additionally a formic acid short term treatment resulted in an effective and economically viable option to recommend. All treatment combinations with biotechnical methods (excluding drone brood removal and hyperthermia) showed on average lower expenses in each year but variability in effectiveness.
