\pdfbookmark[0]{Abstract}{Abstract}
\addcontentsline{toc}{chapter}{Abstract}
\chapter*{Abstract}
\label{sec:abstract}
\vspace*{-10mm}

Context:
Exploratory analysis on the yearly colony loss online survey in Austria over the three years 2018/19, 2019/20 and 2020/21.

Aims:
Main focus is to identify possible reasons for choosing various treatment methods and evaluate efficient and economical of said methods.

Methods:
Analysing results from the questions \enquote{Which main motivations are most likely to apply to you when choosing the Varroa control method (max. 5)?} and \enquote{Estimated treatment expenses per colony without labor cost} in combination with the applied treatment methods and their observed efficiency compared to the observed Austrian average.

Results:
Response rate of 3.5-4\% over the survey years, reported mean expenses on treatment of EUR&nbsp;11.1 and in median EUR&nbsp;9 per colony. Main motivation for choosing a treatment method are 'Efficacy', 'Side effects on bees' and 'Residues'. Rough extrapolation of the varroa agent market shows that the yearly expenses could be in the range of EUR&nbsp;2.9 to 4.7 million a year. Smaller operation ($\leq$ 25 colonies) spend more on treatment per colony than bigger operation. Migratory and Certified Beekeeping operations spend less on average. Predictive ability of the survey question on the treatment expenses is low, with an accuracy on the training dataset of 35\%. Most used combination of formic acid long term treatment in summer followed by oxalic acid trickling in winter showed continuously low mean observed total loss rates, but higher mean expenses in two out of three years. A combination of formic acid short term and oxalic acid sublimation in summer followed by sublimation of oxalic acid in winter showed efficient and economical option. Treatment methods of sublimation of oxalic acid in summer and winter and all treatment combinations with biotechnical methods (excluding drone brood removal and hyperthermia) showed low expenses in each year.

Conclusions:
Motivation shows that beekeepers select methods based on efficiency but also are safe for bees, but also may feel limited by the legal limitations and want to produce residue free produts. Reported expenses per colony show high variance also inside the same treatment methods. Most treatment combinations show in each year overlapping confidence intervals for observed loss rate, indicating possible close same efficiently. Survey treatment methods against varroa is dominated by formic acid application in summer followed by the absolute dominance of oxalic acid treatment in winter.

\vspace*{20mm}

\pdfbookmark[0]{Zusammenfassung}{Zusammenfassung}
\addcontentsline{toc}{chapter}{Zusammenfassung}
{\usekomafont{chapter}Zusammenfassung}
\label{sec:Zusammenfassung}

\blindtext