\pdfbookmark[0]{Abstract}{Abstract}
\addcontentsline{toc}{chapter}{Abstract}
\chapter*{Abstract}
\label{sec:abstract}
\vspace*{-10mm}

Exploratory analysis on the yearly colony loss online survey in Austria over the three winters 2018/19, 2019/20 and 2020/21. Main focus is to identify possible reasons for choosing various treatment methods and evaluate efficiency and expenses of these methods. This is done by analysing the questions \enquote{Which main motivations are most likely to apply to you when choosing the Varroa control method (max. 5)?} and \enquote{Estimated treatment expenses per colony without labour cost} in combination with the applied treatment methods and their efficiency compared to the observed Austrian average. Response rate of 3.5-4.0\% over the survey years, reported mean expenses on treatment of EUR~11.1 and in median EUR~9 per colony. Main motivation for choosing a treatment method are 'Efficacy', 'Side effects on bees' and 'Residues'. Rough extrapolation of the total Austrian varroa agent market shows that the yearly expenses could be in the range of EUR~2.9 to 4.7 million a year. Smaller operation ($\leq$ 25 colonies) spend more on treatment per colony than bigger operation. Migratory and Certified Beekeeping operations spend less on average. Explanatory ability of the survey question on the treatment expenses is low, with an accuracy on the training dataset of 35\%. The most used combination formic acid long term treatment in summer followed by oxalic acid trickling in winter showed consistently low mean observed total loss rates, but higher mean expenses in two out of three years. A combination of formic acid short term in summer and oxalic acid sublimation in summer and winter showed a promise for an efficient and economical treatment. The sublimation of oxalic acid in summer and winter and all treatment combinations with biotechnical methods (excluding drone brood removal and hyperthermia) showed on average lower expenses in each year.

Conclusions:

The main motivations indicate that beekeepers select methods based on efficiency and intrinsics motives for their honeybees well being and are aware of the residue problematic. Smaller hobbyist beekeepers have different motivations than bigger operations. Reported expenses per colony show high variance also inside the same treatment methods. Most treatment combinations show in each year overlapping confidence intervals for observed loss rate, indicating similar efficiency. Survey treatment methods against varroa is dominated by formic acid application in summer followed by the absolute dominance of oxalic acid treatment in winter.

\vspace*{20mm}

\pdfbookmark[0]{Zusammenfassung}{Zusammenfassung}
\addcontentsline{toc}{chapter}{Zusammenfassung}
{\usekomafont{chapter}Zusammenfassung}
\label{sec:Zusammenfassung}