\pdfbookmark[0]{Abstract}{Abstract}
\addcontentsline{toc}{chapter}{Abstract}
\chapter*{Abstract}
\label{sec:abstract}
\vspace*{-15mm}

This work encompasses an exploratory analysis on the annual colony loss survey in Austria over the three winters 2018/19, 2019/20 and 2020/21. The main focus of this study is to identify possible reasons for choosing a treatment method against the parasitic mite \textit{Varroa destructor}. In addition effectiveness and expenses of these methods are evaluated. This is done by analysing the questions \enquote{Which main motivations are most likely to apply to you when choosing the \textit{Varroa} control method (max. 5)?} and \enquote{Estimated treatment expenses per colony without labour cost} in combination with the effectiveness of applied treatment based on observed colony losses. In each survey year around 3.5-4.0\% of Austrian beekeepers participated and answered the online questionnaire. Reported mean expenses on treatments add up to EUR~11.4 and a median of EUR~10.0 per colony. Reported expenses per colony show high variance also within the same treatment methods. Rough extrapolation of the total Austrian \textit{Varroa} mite agent market shows that the yearly expenses are in the range of EUR~2.9 to 4.8 million a year. Main motivation of the participants for choosing a treatment method are 'Efficacy', 'Side effects on bees' and 'Residues'. The main motivations indicate that beekeepers select methods based on efficacy and intrinsic motives for their honey bees well being and are aware of the residue problematic. For bigger operations the motivations 'Residues' and 'Resistances' are a higher motivation than for recreational beekeepers. Most treatment combinations show overlapping confidence intervals for observed loss rates in each year, indicating similar effectiveness. Survey treatment methods against \textit{V. destructor} is dominated by formic acid application in summer followed by the absolute dominance of oxalic acid treatment in winter. Smaller operations ($\leq$ 25 colonies) spend more on treatment per colony than bigger operations. Migratory and certified organic beekeepers spend less on average. A multiple regressions model explanatory ability of treatment methods and operation size on the treatment expenses is low, with an accuracy on the training dataset of 35\%. The most used combination formic acid long term treatment in summer followed by oxalic acid trickling in winter showed low to intermediate loss rates, but high to intermediate expenses compared to other treatment combinations. A combination of formic acid short term treatment with oxalic acid sublimation in summer and winter resulted in an effective and economically viable option that can be recommended. All treatment combinations with 'Another biotechnical method' (excluding drone brood removal and hyperthermia) showed on average lower expenses in each year but variability in effectiveness.
