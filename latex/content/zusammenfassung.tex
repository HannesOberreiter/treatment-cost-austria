\pdfbookmark[0]{Zusammenfassung}{Zusammenfassung}
\addcontentsline{toc}{chapter}{Zusammenfassung}
\chapter*{Zusammenfassung}
\label{sec:Zusammenfassung}
\vspace*{-10mm}

Explorative Analyse der jährlichen Erhebung der Winterverluste von Bienenvölkern in Österreich über die drei Winter 2018/19, 2019/20 und 2020/21. Hauptaugenmerk liegt auf der Identifizierung möglicher Gründe für die Wahl einer Behandlungsmethode gegen die parasitäre Milbe \textit{Varroa destructor}. Darüber hinaus sollen Wirksamkeit und Kosten dieser Methoden bewertet werden. Dies geschieht durch die Analyse der Fragen \enquote{Welche Haupt-Motivationen treffen für Sie bei der Auswahl der Varroabekämpfungsmethode am ehesten zu?} und \enquote{Wie hoch schätzen Sie Ihre Ausgaben für Varroabekämpfungsmittel pro Volk im Zeitraum April bis März ein? (Ohne Arbeitskosten!)} in Kombination mit den angewandten Behandlungsmethoden in Kombination und deren Wirksamkeit im Vergleich zu allen anderen angegebenen Kombinationen. Beteiligungsrate von 3,5-4,0\% der ImkerInnen in den Erhebungsjahren. Durchschnittliche Behandlungskosten von 11,5~Euro und Median 10~Euro pro Bienenvolk. Hohe Varianz der Kosten und diese auch innerhalb derselben Behandlungsmethoden. Eine grobe Hochrechnung des gesamten österreichischen Varroa-Bekämpfungsmittelmarktes zeigt, dass die jährlichen Ausgaben zwischen 2,9 und 4,9 Millionen Euro pro Jahr liegen könnten. Die Hauptgründe für die Wahl einer Behandlungsmethode der TeilnehmerInnen sind "Wirksamkeit", "Nebenwirkungen auf die Bienen" und "Rückstände". Die Hauptmotivationen deuten darauf hin, dass die Imker Methoden aufgrund von Effektivität und intrinsischen Motiven für das Wohlergehen ihrer Honigbienen auswählen und sich der Rückstandsproblematik bewusst sind. Kleinere Hobby-Imker haben andere Beweggründe als größere Betriebe, die eher wirtschaftlich orientiert sind. Die meisten Behandlungskombinationen zeigen in jedem Jahr überlappende Konfidenzintervalle für die beobachteten Verlustraten, was auf eine ähnliche Wirksamkeit hindeutet. Bei der Erhebung der Behandlungsmethoden gegen \textit{Varroa d.} dominiert die Anwendung von der organischen Ameisensäure im Sommer, gefolgt von der absoluten Dominanz der Oxalsäurebehandlung im Winter. Kleinere Betriebe ($\leq$ 25 Bienenvölker) geben mehr für die Behandlung pro Volk aus als größere Betriebe. Wanderimker und zertifizierte Bio-Imker geben im Durchschnitt weniger aus. Ein Regressions Model mit den Behandlungsmethoden und Betriebsgröße als Prädiktoren zum Erklären der Behandlungskosten hatte nur eine geringe Genauigkeit mit 35\% im Trainingsdatensatz. Die am häufigsten verwendete Behandlungskombination aus Ameisensäure-Langzeitbehandlung im Sommer und Oxalsäure-Träufeln im Winter zeigte im Vergleich zu den anderen Behandlungskombinationen niedrige bis mittlere Verlustraten, aber hohe bis mittlere Ausgaben auf. Eine Kombination aus Oxalsäure-Sublimation im Sommer und im Winter und die gleiche, aber zusätzlich eine Ameisensäure-Kurzzeitbehandlung ergab eine wirksame und wirtschaftlich vertretbare Option, die zu empfehlen ist. Alle Behandlungskombinationen mit biotechnischen Methoden (mit Ausnahme der Drohnenbrutentfernung und der Hyperthermie) zeigten im Durchschnitt niedrigere Kosten in jedem Jahr auf, variierten aber in ihrer Wirksamkeit.


