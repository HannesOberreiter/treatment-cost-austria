\pdfbookmark[0]{Zusammenfassung}{Zusammenfassung}
\addcontentsline{toc}{chapter}{Zusammenfassung}
\chapter*{Zusammenfassung}
\label{sec:Zusammenfassung}
\vspace*{-15mm}

Explorative Analyse der jährlichen Erhebung der Winterverluste von Bienenvölkern in Österreich über die drei Winter 2018/19, 2019/20 und 2020/21. Hauptaugenmerk liegt auf der Identifizierung möglicher Gründe für die Wahl einer Behandlungsmethode gegen die parasitäre Milbe \textit{Varroa destructor}. Darüber hinaus sollen Wirksamkeit und Kosten dieser Methoden bewertet werden. Dies geschieht durch die Auswertung der Fragen \enquote{Welche Haupt-Motivationen treffen für Sie bei der Auswahl der \textit{Varroa}-Bekämpfungsmethode am ehesten zu?} und \enquote{Wie hoch schätzen Sie Ihre Ausgaben für \textit{Varroa}-Bekämpfungsmittel pro Volk im Zeitraum April bis März ein? (Ohne Arbeitskosten!)} in Kombination mit den angewandten Behandlungsmethoden und deren Wirksamkeit im Vergleich zueinander. Die Erhebung hat eine Beteiligungsrate von 3,5-4,0\%. Durchschnittliche Behandlungskosten kommen auf 11,4~Euro und Median 10,0~Euro pro Bienenvolk mit einer hohen Varianz und diese auch innerhalb derselben Behandlungsmethoden. Eine Hochrechnung des österreichischen \textit{Varroa}-Bekämpfungsmittelmarktes zeigt, dass die jährlichen Ausgaben zwischen 2,9 und 4,8 Millionen Euro pro Jahr liegen könnten. Die Hauptgründe für die Wahl einer Behandlungsmethode der TeilnehmerInnen sind "Wirksamkeit", "Nebenwirkungen auf die Bienen" und "Rückstände". Dies deutet darauf hin, dass die ImkerInnen aufgrund von Effektivität und intrinsischen Motiven für das Wohlergehen ihrer Honigbienen die Behandlungsmethoden auswählen und sich der Rückstandsproblematik bewusst sind. Für größere Betriebe sind "Rückstände" und "Resistenzen" eine größere Motivation als für Hobby-ImkerInnen. Die meisten Behandlungskombinationen zeigen in jedem Jahr überlappende Konfidenzintervalle für ihren Verlustraten, was auf eine ähnliche Wirksamkeit hindeutet. Bei der Erhebung der Behandlungsmethoden gegen \textit{Varroa d.} dominiert die Anwendung von Ameisensäure im Sommer, gefolgt von der absoluten Dominanz der Oxalsäure im Winter. Kleinere Betriebe ($\leq$ 25 Bienenvölker) geben mehr für die Behandlung pro Volk aus als größere Betriebe. WanderimkerInnen und zertifizierte Bio-ImkerInnen geben im Durchschnitt weniger aus. Ein Regressions Model mit den Behandlungsmethoden und Betriebsgröße als Prädiktoren zum Erklären der Behandlungskosten hatte nur eine geringe Genauigkeit mit 35\% im Trainingsdatensatz. Die am häufigsten verwendete Behandlungskombination aus Ameisensäure-Langzeitbehandlung im Sommer und Oxalsäure-Träufeln im Winter zeigte im Vergleich zu den anderen Kombinationen niedrige bis mittlere Verlustraten, aber hohe bis mittlere Ausgaben auf. Eine Kombination aus Oxalsäure-Sublimation im Sommer und im Winter und die gleiche, aber zusätzlich eine Ameisensäure-Kurzzeitbehandlung resultierte in einer wirksame und wirtschaftlich vertretbare Option, die zu empfehlen ist. Alle Kombinationen mit biotechnischen Methoden (mit Ausnahme der Drohnenbrutentfernung und der Hyperthermie) zeigten im Durchschnitt niedrigere Kosten, variierten aber in ihrer Wirksamkeit.
