\pdfbookmark[0]{Zusammenfassung}{Zusammenfassung}
\addcontentsline{toc}{chapter}{Zusammenfassung}
\chapter*{Zusammenfassung}
\label{sec:Zusammenfassung}
\vspace*{-15mm}

Explorative Analyse der jährlichen Erhebung der Winterverluste von Bienenvölkern in Österreich über die drei Winter 2018/19, 2019/20 und 2020/21. Das Hauptaugenmerk liegt auf der Identifizierung möglicher Gründe für die Wahl einer Bekämpfungsmethode gegen die parasitäre Milbe \textit{Varroa destructor} und die Bewertung der Wirksamkeit und Kosten dieser Methoden. Dies geschieht durch die Auswertung der Fragen \enquote{Welche Haupt-Motivationen treffen für Sie bei der Auswahl der \textit{Varroa}-Bekämpfungsmethode am ehesten zu?} und \enquote{Wie hoch schätzen Sie Ihre Ausgaben für \textit{Varroa}-Bekämpfungsmittel pro Volk im Zeitraum April bis März ein? (Ohne Arbeitskosten!)} in Kombination mit den angewandten Methoden und der beobachteten Winterverluste. Der Online Umfragebogen hat eine Beteiligungsrate von 3,5-4,0\% der österreichischen ImkerInnen. Durchschnittliche Bekämpfungskosten betragen 11,4~Euro und ergebenen einen Median von 10,0~Euro pro Volk mit einer hohen Varianz, auch innerhalb derselben Methode. Eine Hochrechnung des österreichischen \textit{Varroa}-Bekämpfungsmittelmarktes zeigt, dass die jährlichen Ausgaben zwischen 2,9 und 4,8 Millionen Euro pro Jahr liegen. Die Hauptgründe für die Wahl einer Bekämpfungsmethode sind "Wirksamkeit", "Nebenwirkungen auf die Bienen" und "Rückstände". Dies deutet darauf hin, dass die ImkerInnen aufgrund von Effektivität und intrinsischen Motiven für das Wohlergehen ihrer Honigbienen die Methoden auswählen und sich der Rückstandsproblematik bewusst sind. Für größere Betriebe sind "Rückstände" und "Resistenzen" eine größere Motivation als für Freizeit-ImkerInnen. Die meisten Bekämpfungsmethoden zeigen in jedem Jahr überlappende Konfidenzintervalle für ihren Verlustraten, was auf eine ähnliche Wirksamkeit hindeutet. Bei der Erhebung der Bekämpfungsmethoden gegen \textit{V. destructor} dominiert die Anwendung von Ameisensäure im Sommer, gefolgt von der Oxalsäure im Winter. Kleinere Betriebe ($\leq$ 25 Bienenvölker) geben mehr für die Behandlung pro Volk aus als größere Betriebe. WanderimkerInnen und zertifizierte Bio-ImkerInnen geben im Durchschnitt weniger aus. Ein Regressionsmodell mit den Behandlungsmethoden und der Betriebsgröße als Prädiktoren zum Erklären der Kosten hatte nur eine geringe Genauigkeit mit 35\% im Trainingsdatensatz. Die am häufigsten verwendete Bekämpfungsmethoden-Kombination, Ameisensäure-Langzeitbehandlung im Sommer und Oxalsäure-Träufeln im Winter, zeigte im Vergleich zu den anderen Kombinationen niedrige bis mittlere Verlustraten, aber hohe bis mittlere Kosten auf. Eine Kombination aus Ameisensäure-Kurzzeitbehandlung und Oxalsäure-Sublimation im Sommer und im Winter resultierte in einer wirksamen und wirtschaftlich vertretbaren Option, die zu empfehlen ist. Alle Kombinationen mit biotechnischen Methoden (mit Ausnahme der Drohnenbrutentfernung und der Hyperthermie) zeigten im Durchschnitt niedrigere Kosten, variierten aber in ihrer beobachteten Wirksamkeit.
