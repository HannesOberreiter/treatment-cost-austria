\pdfbookmark[0]{Preface}{Preface}
\addcontentsline{toc}{chapter}{Preface}
\chapter*{Preface}
\label{sec:Preface}
\vspace*{-10mm}

Following work represents my master thesis as partial fulfilment for my Master's degree at the University of Graz. The technical progress in our lifetime is happening unimaginably fast and computer science has already become an indispensable part of the natural sciences. Since I have dealt intensively with various programming languages and am technically versed, I was able to expand my knowledge from my Bachelor's degree in biology through applied, computer-based working methods. The focus of my individual Master's degree \enquote{Computer science in ecology and evolutionary biology}, is on modern technologies and statistics in order to direct my existing knowledge of biological processes in a modern and forward-looking direction and to be able to process \enquote{big data} with the help of bioinformatics. Nevertheless, it combines these modern aspects of science with empirical research methods in order to be able to work on biological questions from sample collection to various modern methods of analysis.

This Master thesis was done in mind with my goals from my individual Master curriculum. it allowed me to practice analysis of data, applying of different statistics and graphical visualization of results and ultimately infer possible answers to ecological questions. The foundation of this work is the data collected by a citizen science survey of the yearly managed bee colony overwinter losses in Austria. I did work with this data already intensively as part of my bachelor degree in Biology and was further involved in the project \enquote{Zukunft Biene 2} were this survey is part of. This means I already have a good and deep knowledge about the data at hand. The analysis for my master thesis includes the three survey years 2018/19, 2019/20 and 2020/21. The difference to the previous work I did with the data is that this time I wanted to generate a bridge between ecological analysis and a more economic driven thematic. The main questions of interest is the expenses per colony to treat against the aggressor \textit{Varroa destructor}. The work was meant to be helpful for beekeepers in further decision making which treatment is not only effective but also keeps the expenses in mind. Another big point on my list was that the analysis is reproducibility in the mindset of open source, therefore I publish all of my cleaned and commented code in a GitHub repository. This should help to understand the approach and gives a good starting point for other studies and countries who want to carry out a similar study.