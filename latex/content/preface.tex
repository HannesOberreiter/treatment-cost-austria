\pdfbookmark[0]{Preface}{Preface}
\addcontentsline{toc}{chapter}{Preface}
\chapter*{Preface}
\label{sec:Preface}
\vspace*{-10mm}

The technical progress in our lifetime is happening unimaginably fast, and computer science has already become an indispensable part of natural science. Since I have dealt with various programming languages and I am technically versed, I was able to expand my knowledge from my Bachelor's degree in biology through applied, computer-based working methods. The focus of my individual Master's degree \enquote{Computer science in ecology and evolutionary biology}, is on modern technologies and statistics to direct my existing knowledge of biological processes in a modern and forward-looking direction and to be able to process \enquote{big data} with the help of bioinformatics. The combination of computer science, modern tools and empirical research methods allows me to tackle on biological questions in ecology and evolutionary biology in novel ways. One of these questions represent the following work, which is my thesis as partial fulfilment for my Master's degree at the University of Graz. The foundation of this work is the data collected by a yearly citizen science survey on managed honey bee winter colony losses in Austria. I did work with this data already intensively as part of my bachelor degree and was further involved in the project \enquote{Zukunft Biene 2} were this survey was part of. This means I already possess an excellent knowledge about the data at hand. The difference to the previous work I did with the data is that this time I wanted to generate a bridge between ecological analysis and a more economic driven thematic. The key question of interest is how much money beekeepers spend to treat against the mite \textit{Varroa destructor} and a categorical question why they did chose the method. The work is intended to be helpful for beekeepers in their decision making and give new insight into the beliefs of Austrian beekeepers. Another principal point on my list was that the analysis was reproducible and open source\footnote{\url{https://github.com/HannesOberreiter/treatment-expenses-austria}}. This should help to follow the approach and gives an excellent starting point for other studies who want to carry out a similar research. Although, the thesis does not strictly answer any ecological or evolutionary question as one would expect based on my individual master curriculum title. It allowed me to practice the analysis of data, applying of different statistics and graphical visualization of results. Followed by a desk and literature research and finally writing a work which I believe follows the scientific community best practices. Therefore, the thesis did allow me to acquire a good and handy skill set, which can be applied in a broad range of biological associated fields. Hopefully, in my future I get the chance to use these skills and keep combining computer science with biology and may make a small healthy and sustainable footprint on earth.